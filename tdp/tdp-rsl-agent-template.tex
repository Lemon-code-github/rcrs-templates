\documentclass[runningheads,a4paper]{llncs}
%%
\usepackage{amsmath,amsfonts,amssymb}
\usepackage{graphicx}
\usepackage[utf8]{inputenc}
\usepackage[hidelinks]{hyperref}
\usepackage{url}
\usepackage{float}
\usepackage{amsmath}
\usepackage{graphicx}
\usepackage{subfig}
\usepackage{wrapfig}
\usepackage{comment}
\usepackage{multirow}
\usepackage{adjustbox}
%%
%% Figure positioning
\setlength{\intextsep}{10pt plus 2pt minus 2pt}
\usepackage[belowskip=-1pt,aboveskip=5pt]{caption}
%%
\begin{document}
%%
%% Title
\title{RoboCup Rescue YEAR\\
       TDP Agent Simulation\\
       TEAM-NAME (COUNTRY)}
%%
%% Authors
\author{Author 1\thanks{Corresponding author.} \and Author 2 \and Author 3}
\institute{Affiliation, Country \\
           \texttt{[author1.email, author2.email, author3.email] (optional)}\\
           \url{http://web-site.url} \texttt{(optional)}}
%%
\maketitle
%%
\begin{abstract}
%%
The abstract should summarize the problem and hypotheses tackled by, the methods used by the team, as well as the results and conclusions achieved.
%%
\end{abstract}
%%

%%%%%%%%%%%%%%%%%%%%%%%%%%%%%%%%%%%%%%%%%%%%%%%%%%%%%%%%%%%%%%%%%%%%%%%%%%%%%%%%
%% Remove this section before submitting the TDP
%%%%%%%%%%%%%%%%%%%%%%%%%%%%%%%%%%%%%%%%%%%%%%%%%%%%%%%%%%%%%%%%%%%%%%%%%%%%%%%%
\section*{Very Important Notes}
%%
Each team is required to write their TDP according to the section structure provided in this template. If the team has any question about the require TDP content, send an email to \href{mailto:robocuprescuesim@gmail.com}{robocuprescuesim@gmail.com}.

Replace the YEAR, TEAM-NAME and COUNTRY terms in the title with the team and competition information.

Identify one of the authors as the corresponding author and make sure that this author's email address is correctly provided.

In Section~\ref{sec:intro}, teams are required to provide a clear description of the main contribution and approach used to implement their agent team.

\textbf{IMPORTANT}: Teams are required to reference the scientific relevance of their work and cite the high-quality scientific articles related to their work. If the team was motivated, based on, or extended on an approach previously proposed or used by another team, do not forget to give credit by citing the work.

In Section~\ref{sec:modules}, the team must describe how each of the ADF modules were implemented to fulfill the approach briefly described in Section~\ref{sec:intro}. Each module should describe in general terms the issue tackled by the module, previous approaches to solve the issue, and how the team implemented it, and highlight the implementation advantages and disadvantages. Each module description should be structured with the subsections \texttt{Purpose}, \texttt{Related Works}, \texttt{Proposed Approach}, and \texttt{Pros and Cons}, please see an example in Section~\ref{sec:clustering}.

In Section~\ref{sec:strategies}, the team must describe how each of the agents was implemented to fulfill the approach briefly described in Section~\ref{sec:intro}. Each agent description should be structured with the subsections \texttt{Purpose}, \texttt{Related Works}, \texttt{Proposed Approach}, and \texttt{Pros and Cons}, please see an example in Section~\ref{sec:clustering}.

Please use BibTeX for formatting the list of references in the \texttt{references.bib} file and use the macro \texttt{\textbackslash cite\{ref1\}} for citing these references in the TDP, for example~\cite{ref1}.

Be care about plagiarism, if a team commits plagiarism, the team and its members will be banned participating on the current and next year's RoboCup Rescue Simulation competitions. The term plagiarism comprises any use of external knowledge without proper referencing, i.e., copying or using thoughts, ideas, texts or language in general and presenting them as their own. This applies for Team Description Papers as well as team code. All kinds of licenses and copyright have to be respected. This applies to the qualification process and the RoboCup tournaments. Please be aware that when a team is found guilty of committing plagiarism it is disqualified and banned at any time. This may also be in the middle of the tournament.
%%%%%%%%%%%%%%%%%%%%%%%%%%%%%%%%%%%%%%%%%%%%%%%%%%%%%%%%%%%%%%%%%%%%%%%%%%%%%%%%
%%
\section{Introduction}
\label{sec:intro}
%%
The introduction should highlight the main issues approached by the team. The team should provide a brief overview of these issues and which ADF modules were mostly altered to tackle these issues. It is also extremely important to describe the main improvements made by the team since last participation in the RoboCup Rescue Agent competition, if applies.
%%
\section{Modules}
\label{sec:modules}
%%
In this section, the team should describe the features and possible improvements implemented in the main ADF clustering, path planning and communication modules. The team should remove the subsections that do not apply.
%%
\subsection{Clustering}
\label{sec:clustering}
%%
Clustering ADF module implementation details.
%%
\subsubsection{Purpose}
\label{sec:clustering:purpose}
%%
Please describe briefly the main purpose of this module using a general description without any rescue simulation concepts. This description should include the essential concepts take into account to solve this task.
%%
\subsubsection{Related Works}
\label{sec:clustering:related}
%%
Provide a summary of the related scientific works that approaches this task even though they have not ever been used in the Rescue Agent Simulation competition.

Describe briefly how other teams have approached this task and summarize their strengths and weaknesses.

Do not forget to properly cite the reviewed work.
%%
\subsubsection{Proposed Approach}
\label{sec:clustering:proposed}
%%
Describe the proposed approach to tackle this task explaining it in detail.
%%
\subsubsection{Pros and Cons}
\label{sec:clustering:pros-cons}
%%
Identify the strengths and weaknesses of the proposed approach.
%%
\subsection{Path Planning}
\label{sec:path}
%%
Path planning ADF module implementation details. Please structure this section according to the structure of Section~\ref{sec:clustering}.
%%
\subsection{Communication}
\label{sec:comm}
%%
Communication ADF module implementation details. Please structure this section according to the structure of Section~\ref{sec:clustering}.
%%
\section{Strategies}
\label{sec:strategies}
%%
Describe the features of the ADF modules implemented specifically to each type of agent and their main agent strategies. 
%%
\subsection{Police Force}
\label{sec:policeForce}
%%
Describe the Police Force and Police Office agent details. Please structure this section according to the structure of Section~\ref{sec:clustering}.
%%
\subsection{Ambulance Team}
\label{sec:ambulanceTeam}
%%
Describe the Ambulance Team and Ambulance Center agent details. Please structure this section according to the structure of Section~\ref{sec:clustering}.
%%
\subsection{Fire Brigade}
\label{sec:fireBrigade}
%%
Describe the Fire Brigade and Fire Station agent details. Please structure this section according to the structure of Section~\ref{sec:clustering}.
%%
\section{Preliminary Results}
\label{sec:results}
%%
It is highly advised to present some results of the team developed using the maps from the previous year and compare their results.
%%
\begin{table}
  \centering
  \begin{tabular}{llll}
    \hline
    \multirow{2}{*}{\textbf{Team}}  & \multicolumn{3}{c}{\textbf{Map}}\\
    \cline{2-4}
                                    & \textbf{Kobe} & \textbf{Paris}  & \textbf{Berlin}\\
    \hline
    My Team                         & 44.2          & 77.12           & 23.56\\
    Team A                          & 30.34         & 56.78           & 1.23\\
    Team B                          & 45.7          & 11.8            & 70.2\\
    \hline
  \end{tabular}
\end{table}
%%
\section{Conclusions}
\label{sec:conclusions}
%%
Summarize the main contributions and conclusions, and provide some ideas for
future improvements.
%%
\bibliographystyle{splncs04}
\bibliography{references}
%%
\end{document}
%%