\documentclass[runningheads,a4paper]{llncs}
%%
\usepackage{amsmath,amsfonts,amssymb}
\usepackage{graphicx}
\usepackage[utf8]{inputenc}
\usepackage[hidelinks]{hyperref}
\usepackage{url}
\usepackage{float}
\usepackage{amsmath}
\usepackage{graphicx}
\usepackage{subfig}
\usepackage{wrapfig}
\usepackage{comment}
\usepackage{multirow}
\usepackage{adjustbox}
%%
%% Figure positioning
\setlength{\intextsep}{10pt plus 2pt minus 2pt}
\usepackage[belowskip=-1pt,aboveskip=5pt]{caption}
%%
\begin{document}
%%
%% Title
\title{RoboCup 2018 -- TDP Rescue Agent Simulation\\
       Team Name (Country)}
%%
%% Authors
\author{Author 1 \and Author 2 \and Author 3}
\institute{Affiliation, Country \\
           \texttt{[author1.email, author2.email, author3.email] (optional)}\\
           \url{http://web-site.url} \texttt{(optional)}}
%%
\maketitle
%%
\begin{abstract}
%%
The abstract should summarize the problem, hypotheses, methods used, results, and conclusions.
%%
\end{abstract}
%%
\section{Introduction}
%%
The introduction should highlight the main issues approached by the team. The 
team should provide a brief overview of these issues and highlight other 
approaches to solve them. It is also extremely important to describe the main 
improvements made by the team since last year, if the team participated 
previously in the RoboCup Rescue Agent competition.
%%
\section{Modules}
%%
In this section, the team should describe the features and improvement
implemented in the main ADF modules used by all agents like clustering, path
planning and communication modules.
%%
\subsection{Clustering}
%%
Clustering module details.
%%
\subsection{Path Planning}
%%
Path Planning module details.
%%
\subsection{Communication}
%%
Communication module details.
%%
\section{Strategies}
%%
In this section, the team should describe the features of the ADF modules
implemented specifically to each type of agent and their main strategies. The
team should remove the subsections that does not apply.
%%
\subsection{Police Office}
%%
Describe the police office target allocation implementation and strategy.
%%
\subsection{Police Force}
%%
Describe the police force target allocation implementation and strategy.
%%
\subsection{Ambulance Center}
%%
Describe the ambulance center target allocation implementation and strategy.
%%
\subsection{Ambulance Team}
%%
Describe the ambulance team target allocation implementation and strategy.
%%
\subsection{Fire Station}
%%
Describe the fire station target allocation implementation and strategy.
%%
\subsection{Fire Brigade}
%%
Describe the fire brigade target allocation implementation and strategy.
%%
\section{Preliminary Results}
It is highly advised to present some results of the new algorithms and
strategies developed by the team using the maps from the previous year and
compare the results with last year's results.
%%
\begin{table}
  \begin{tabular}{llll}
    \hline
    \multirow{2}{*}{\textbf{Team}}  & \multicolumn{3}{c}{\textbf{Map}}\\
    \cline{2-4}
                                    & \textbf{Kobe} & \textbf{Paris}  & \textbf{Berlin}\\
    \hline
    My Team                         & 44.2          & 77.12           & 23.56\\
    Team A                          & 30.34         & 56.78           & 1.23\\
    Team B                          & 45.7          & 11.8            & 70.2\\
    \hline
  \end{tabular}
\end{table}
%%
\section{Conclusions}
%%
Summarize the main contributions and conclusions, and provide some ideas for
future improvements.
%%
\bibliographystyle{splncs03}
\bibliography{references}
%%
\end{document}
%%